\documentclass[aspectratio=169,handout]{beamer}
% \usepackage[utf8]{inputenc}
\usetheme{metropolis}
\usecolortheme{orchid}
\usepackage{amsmath}
\usepackage{amssymb}
\usepackage{amsthm}
\usepackage{multirow}
\usepackage[ruled]{algorithm2e}
\usepackage{mathtools}
\usepackage{caption}
% \usepackage{epstopdf}
\usepackage{hyperref}
\usepackage{subcaption}
\usepackage{siunitx}
\usepackage{cancel}
\usepackage{textcomp}
\usepackage{tcolorbox}
\usepackage{tikz}
\usetikzlibrary{tikzmark,positioning,arrows.meta,shapes.misc,calc}

\setbeamerfont{footnote}{size=\tiny}

% Information boxes
\newcommand*{\info}[4][16.3]{%
  \node [ annotation, #3, scale=0.65, text width = #1em,
          inner sep = 2mm ] at (#2) {%
  \list{$\bullet$}{\topsep=0pt\itemsep=0pt\parsep=0pt
    \parskip=0pt\labelwidth=8pt\leftmargin=8pt
    \itemindent=0pt\labelsep=2pt}%
    #4
  \endlist
  };
}

%% Definitions for root locus plots
\newcommand*{\rootlocusexample}[4]{% xmin,xmax,ymin,ymax,
		\foreach \x in {#1,...,#2}{
			\ifthenelse{\x=0}{}
		  {% false case
   			\draw (\x cm,1pt) -- (\x cm,-1pt) node[anchor=north] {$\x$};
		  }
		}
		\foreach \y in {#3,...,#4}{
			\ifthenelse{\y=0}{}
		  {% false case
   			\draw (-1pt,\y cm) -- (1pt,\y cm) node[anchor=east] {$\y$};
		  }
   	}
		\draw [-latex] (#1,0) -- (#2,0) node [above]  {$\sigma$};
		\draw [-latex] (0,#3) -- (0,#4) node [right] {$j\omega$};
}

\def\centerarc[#1](#2)(#3:#4:#5)% Syntax: [draw options] (center) (initial angle:final angle:radius)
    { \draw[#1] ($(#2)+({#5*cos(#3)},{#5*sin(#3)})$) arc (#3:#4:#5); }

\tikzset{%
  >={Latex[width=2mm,length=2mm]},
  % Specifications for style of nodes:
            base/.style = {rectangle, rounded corners, draw=black,
                           minimum width=4cm, minimum height=1cm,
                           text centered, font=\sffamily},
  activityStarts/.style = {base, fill=blue!30},
       startstop/.style = {base, fill=red!30},
    activityRuns/.style = {base, fill=green!30},
         process/.style = {base, minimum width=2.5cm, fill=orange!15,
                           font=\ttfamily},
}

\tikzset{pole/.style={cross out, draw=black, minimum size=2*(#1-\pgflinewidth), inner sep=0pt, outer sep=0pt},
pole/.default={3pt}}
\tikzset{poledes/.style={rectangle, draw=black, minimum size=2*(#1-\pgflinewidth), inner sep=0pt, outer sep=0pt},
poledes/.default={3pt}}
\tikzset{zero/.style={circle, draw=black, fill=white, minimum size=2*(#1-\pgflinewidth), inner sep=0pt, outer sep=0pt},
zero/.default={3pt}}
\tikzset{test/.style={rectangle, draw=black, fill=white, minimum size=2*(#1-\pgflinewidth), inner sep=0pt, outer sep=0pt},
test/.default={3pt}}

%% Definitions for block diagrams
\tikzstyle{block} = [draw, fill=blue!20, rectangle, 
    minimum height=2em, minimum width=3em]
\tikzstyle{sum} = [draw, fill=blue!20, circle, node distance=1cm]
\tikzstyle{input} = [coordinate]
\tikzstyle{output} = [coordinate]
\tikzstyle{pinstyle} = [pin edge={to-,thin,black}]

\renewcommand\textbullet{\ensuremath{\bullet}}
\newcommand\scalemath[2]{\scalebox{#1}{\mbox{\ensuremath{\displaystyle #2}}}}
\newcommand{\norm}[1]{\left\lVert#1\right\rVert}

%%% Bibliography
\usepackage[citestyle=numeric,style=numeric,backend=biber,doi=false,isbn=false,url=false]{biblatex}
\addbibresource{references.bib}

%%% Suppress biblatex annoying warning
\usepackage{silence}
\WarningFilter{biblatex}{Patching footnotes failed}

%%% new theorems %%%%%%%%%%%%%%%%%%%%%%%%%%%%%%%%%%%%%%%%%%%%%%%%%%%%%%%%%%%%%%
\theoremstyle{definition}
\newtheorem{mydef}{Definition}

\theoremstyle{plain}
\newtheorem{mylemma}{Lemma}[section]
\newtheorem{mytheorem}{Theorem}[section]
\newtheorem{myproposition}{Proposition}[section]
\newtheorem{myproblem}{Problem}[section]
\newtheorem{mydefinition}{Definition}[section]
\newtheorem{myassumption}{Assumption}[section]

\theoremstyle{remark}
\newtheorem{myremark}{Remark}[section]

\newcounter{saveenumi}
\newcommand{\seti}{\setcounter{saveenumi}{\value{enumi}}}
\newcommand{\conti}{\setcounter{enumi}{\value{saveenumi}}}

\resetcounteronoverlays{saveenumi}

\title{Controles}
\subtitle{\small Clase 8: Respuesta en Frecuencia - Compensadores en Atraso, Atraso-Adelanto}
\author{Gerardo Becerra, Ph.D.}
\institute{Pontificia Universidad Javeriana\\ Departamento de Electrónica}
\date{}
 
\begin{document}

\frame{\titlepage}	

\begin{frame}[<+->]\frametitle{Compensadores - Características de los Diferentes Esquemas de Compensación}
 	\begin{itemize}
 		\item Compensación en adelanto:
 		\begin{itemize} 
 			\item Ofrece una mejora notable en la respuesta transitoria con un pequeño cambio en la exactitud de estado estacionario.
 			\item Puede acentuar efectos de ruido de alta frecuencia.
 			\item Aumenta el orden del sistema en 1 (a menos que exista cancelación polo-cero).
 		\end{itemize}
 		\item Compensación en atraso:
 		\begin{itemize}
 			\item Ofrece una mejora notable en la exactitud de estado estacionario, pero aumentando el tiempo de la respuesta transitoria.
 			\item Elimina los efectos de señales de ruido de alta frecuencia.
 			\item Aumenta el orden del sistema en 1 (a menos que exista cancelación polo-cero).
 		\end{itemize}
 		\item Compensación adelanto-atraso:
 		\begin{itemize}
 			\item Combina las características de los dos tipos de compensación.
 			\item Aumenta el orden del sistema en 2 (a menos que exista cancelación polo-cero).
 			\item El sistema se hace más complejo y es más difícil controlar la respuesta transitoria.
 		\end{itemize}
  \end{itemize}    
\end{frame}

\section{Compensación en Atraso Usando la Respuesta en Frecuencia}
\begin{frame}[<+->]\frametitle{Compensador en Atraso}
	\begin{equation*}
		G_c(s) = K_c \beta \frac{T s + 1}{\beta T s + 1} = K_c \frac{s+\frac{1}{T}}{s + \frac{1}{\beta T}}\hspace*{6mm} (\beta > 1)
	\end{equation*}
	\pause
	\begin{itemize}
		% \item $\beta$: Factor de atenuación del compensador.
		\item El compensador tiene un cero en $s = -1/T$ y un polo en $s = -1/(\beta T)$.
		\item Dado que $\beta > 1$ $\Rightarrow$ el polo se ubica a la derecha del cero.
	\end{itemize}
\end{frame}

\begin{frame}[c]\frametitle{Compensador en Atraso}
\begin{columns}
	\begin{column}{0.4\textwidth}
	\small
	Diagrama de Bode del compensador para $K_c=1$, $\beta=10$.
	\begin{itemize}
		\item<2-> Frecuencias de quiebre: $\omega = 1/T$, $\omega = 1/(\beta T)$.
		\item<3-> Frecuencias bajas: $|G_c(s)| = 20$ dB.
		\item<4-> Frecuencias altas: $|G_c(s)| = 0$ dB.
		\item<5-> Compensador en atraso $\Leftrightarrow$ Filtro pasabajos.
	\end{itemize}
	\end{column}
	\begin{column}{0.6\textwidth}
	\begin{figure}
		\centering
		\includegraphics<1->[width=8cm]{images/bodeLagComp.eps}
	\end{figure}
	\end{column}
\end{columns}
\end{frame}

\begin{frame}[c]\frametitle{Compensador en Atraso: Procedimiento de Diseño}
	\begin{enumerate}
		\item<1-> Se asume la siguiente forma para el compensador en atraso:
		\begin{equation*}
			G_c(s) = K_c \beta \frac{Ts+1}{\beta Ts+1} = K_c \frac{s+\frac{1}{T}}{s+\frac{1}{\beta T}}, \hspace*{3mm} (\beta > 1)
		\end{equation*}
		\pause
		Definiendo $K_c \beta = K$, se tiene
		\begin{equation*}
			G_c(s) = K \frac{Ts+1}{\beta Ts+1}
		\end{equation*}
		\pause
		La F.T. de lazo abierto del sistema compensado es:
		\begin{equation*}
			G_c(s)G(s) = K \frac{Ts+1}{\beta Ts+1}G(s) = \frac{Ts+1}{\beta Ts+1} KG(s) = \frac{Ts+1}{\beta Ts+1} G_1(s)
		\end{equation*}
		donde $G_1(s) = KG(s)$.
		\pause
		Determinar $K$ para satisfacer el requerimiento dado por la constante de error estático de velocidad.
		\seti
	\end{enumerate}
\end{frame}

\begin{frame}[<+->]\frametitle{Compensador en Atraso: Procedimiento de Diseño}
	\begin{enumerate}
		\conti
		\item Determine el margen de fase del sistema $G_1(s)$, y si es insuficiente continúe con los siguientes pasos.
		\item Determine la frecuencia $\omega_c'$ donde el requerimiento de margen de fase sería satisfecho si la curva de magnitud cruzara por 0 dB en ésta frecuencia. Incluya un márgen adicional de seguridad de \ang{5}.
		\item Ubique el cero del compensador una década por debajo de la nueva frecuencia de cruce de ganancia: $\omega_z = \omega_c'/10$.
		\seti
	\end{enumerate}
\end{frame}

\begin{frame}[<+->]\frametitle{Compensador en Atraso: Procedimiento de Diseño}
	\begin{enumerate}
		\conti
		\item Calcule la atenuación necesaria en $\omega_c'$ para asegurar que la curva de magnitud cruza por cero.
		\item Calcule el valor de $\beta$ notando que la atenuación introducida por el compensador en $\omega_c'$ es igual a $20 \log \left(\frac{1}{\beta} \right)$.
		\item Calcule la ubicación del polo como $\omega_p = \frac{1}{\beta T} = \frac{\omega_z}{\beta}$.
		\item Verifique si el sistema de control diseñado satisface los requerimientos. En caso contrario realizar nuevamente la ubicación del polo y cero hasta lograr un resultado satisfactorio.
		\seti
	\end{enumerate}
\end{frame}

\begin{frame}[<+->]\frametitle{Compensador en Atraso: Ejemplo}
	Considere el sistema $G(s) = \frac{1}{s(s+1)(0.5s+1)}$ con retroalimentación unitaria:
	\begin{figure}
		\centering
		\includegraphics[width=5cm]{images/lagCompensatorExampleSystem.eps}
	\end{figure}
	Se desea diseñar un compensador para que el sistema tenga una constante de error estático de velocidad $K_v = 5$, margen de fase de al menos \ang{40} y margen de ganancia de al menos 10 dB.
\end{frame}

\begin{frame}[<+->]\frametitle{Compensador en Atraso: Ejemplo}
	En primer lugar se calcula $K$ para satisfacer el requerimiento de $K_v = 5$:
	\begin{align*}
		\onslide<1->{K_v &= 5 = \lim_{s\rightarrow 0} sG_c(s)G(s)} 
		\onslide<2->{= \lim_{s \rightarrow 0} \cancel{s}K_c\beta\frac{Ts+1}{\beta T s + 1} \frac{1}{\cancel{s}(s+1)(0.5s+1)}}
		\onslide<3->{= K_c\beta\\}
		\onslide<4->{& \text{Asumiendo } K_c\beta = K \Longrightarrow K = 5}
	\end{align*}
	\onslide<5->{Con $K = 5$, el sistema satisface el requerimiento de desempeño. Entonces, la función de transferencia de lazo abierto del sistema no compensado con ganancia ajustada queda:
	\begin{equation*}
		G_1(j\omega) = \frac{5}{j\omega(j\omega+1)(0.5j\omega+1)}
	\end{equation*}}
\end{frame}

\begin{frame}[c]\frametitle{Compensador en Atraso: Ejemplo}
  \begin{columns}
  	\begin{column}{0.3\textwidth}
			\onslide<1->{Usando la función \texttt{margin} de Matlab para $G_1(s)$ se obtiene:\\}
			\onslide<2->{\textcolor{red}{\small PM = -\ang{13} en 1.8 rad/s}\\}
			\onslide<3->{\textcolor{green}{\small GM = -4.44 dB en 1.41 rad/s}\\}
			\onslide<4->{Note que el sistema no compensado con ganancia ajustada en lazo cerrado es \textbf{inestable}!}
  	\end{column}
  	\begin{column}{0.7\textwidth}
			\begin{figure}
				\centering
				\includegraphics<1->[width=9cm]{images/bodeGainAdjustedSystem.eps} 	
		  \end{figure}
  	\end{column}
  \end{columns}
\end{frame}

\begin{frame}[c]\frametitle{Compensador en Atraso: Ejemplo}
	\vspace*{5mm}
	\begin{columns}
		\begin{column}{0.4\textwidth}
			\small
			\onslide<1->{Para satisfacer el margen de fase deseado de \ang{40} se identifica la frecuencia $\omega_c'$ para la cual la fase faltante $\phi_m$ para alcanzar -\ang{180} es igual a \ang{40} más un pequeño margen de \ang{5}.\\}
			\vspace*{2mm}
			\onslide<2->{Del diagrama de Bode se observa que en $\omega_c' = 0.56$ rad/s la fase es -\ang{135}, es decir $-\ang{135} - \phi_m = -\ang{180}$.\\}
			\vspace*{2mm}
			\onslide<3->{$\omega_c'$ se toma como la nueva frecuencia de cruce de ganancia para el sistema compensado.}
		\end{column}		
		\begin{column}{0.6\textwidth}
				\centering
				\includegraphics<2->[width=8.5cm]{images/bodeCrossoverFrequency.eps} 	
		\end{column}		
	\end{columns}
\end{frame}

\begin{frame}[c]\frametitle{Compensador en Atraso: Ejemplo}
	\vspace*{5mm}
	\begin{columns}
		\begin{column}{0.4\textwidth}
			\small
			\onslide<1->{Ubicando el cero del compensador una década por debajo de la frecuencia de cruce de ganancia $\omega_c'$ se obtiene $\omega_z = 0.056$ rad/s.\\}
			\vspace*{2mm}
			\onslide<2->{Calculando la atenuación necesaria en $\omega_c'$ para garantizar el cruce por cero de la ganancia se tiene que:}
			\begin{align*}
				\onslide<3->{&20 \log \left(\frac{1}{\beta}\right) = -17.51\\}
				\onslide<4->{& \Longrightarrow \beta = 7.5076}
			\end{align*}
		\end{column}		
		\begin{column}{0.6\textwidth}
				\centering
				\includegraphics<1->[width=8.5cm]{images/bodeCrossoverFrequency.eps} 	
		\end{column}		
	\end{columns}
\end{frame}

\begin{frame}[c]\frametitle{Compensador en Atraso: Ejemplo}
	\begin{columns}
	\begin{column}{0.5\textwidth}
		\onslide<1->{La ubicación del polo se obtiene como
		\begin{equation*}
			\omega_p = \frac{\omega_z}{\beta} = 0.0075
		\end{equation*}}
		\onslide<2->{El parámetro $T$ se obtiene como:
		\begin{equation*}
			T = \frac{1}{\beta \omega_p} = 17.8571
		\end{equation*}}
	\end{column}	
	\begin{column}{0.5\textwidth}
		\onslide<3->{Finalmente, se calcula $K_c$ como:
		\begin{equation*}
			K_c = \frac{K}{\beta} = 0.666
		\end{equation*}}
		\onslide<4->{Entonces, el compensador obtenido es:
		\begin{equation*}
			G_c(s) = 5 \frac{1+17.8571s}{1+134.0639s}
		\end{equation*}}
	\end{column}	
	\end{columns}
\end{frame}


\begin{frame}[c]\frametitle{Compensador en Atraso: Ejemplo - Diagrama de Bode}
\begin{columns}
	\begin{column}{0.35\textwidth}
		\small
		\onslide<1->{Los márgenes obtenidos para el sistema compensado son:\\}
		\vspace*{2mm}
		\onslide<2->{\textcolor{red}{PM = \ang{40.1} en 0.562 rad/s.}\\}
		\vspace*{2mm}
		\onslide<3->{\textcolor{green}{GM = 12.4 dB en 1.36 rad/s.}\\}
		\vspace*{2mm}
		\onslide<4->{Se redujo el ancho de banda del sistema $\rightarrow$ filtro pasabajos.\\}
		\vspace*{2mm}
		\onslide<5->{El compensador modifica la respuesta para bajas frecuencias.\\}
		\vspace*{2mm}
		\onslide<6->{La atenuación en altas frecuencias mejora el margen de fase.}
	\end{column}
	\begin{column}{0.65\textwidth}
		\begin{figure}
			\centering
			\includegraphics<1->[width=9cm]{images/bodeLagCompExampleComparison.eps}
		\end{figure}
	\end{column}
\end{columns}
\end{frame}

\begin{frame}[c]\frametitle{Compensador en Atraso: Ejemplo - Respuesta Paso Sistema Compensado}
\begin{columns}
	\begin{column}{0.3\textwidth}
		\small
		\onslide<2->{El sistema compensado en lazo cerrado presenta una respuesta más lenta $\Leftarrow$ menor ancho de banda.\\}
		\vspace*{2mm}
		\onslide<3->{Polos de lazo cerrado: \scriptsize \{$-0.2776 \pm 0.6629j$, $-2.3919$, $-0.0604$\}.\\}
		\vspace*{2mm}
		\small
		\onslide<4->{Polo dominante muy cercano al eje $j\omega$ $\Rightarrow$ cola de baja amplitud y lenta caída.}
	\end{column}
	\begin{column}{0.7\textwidth}
		\begin{figure}
			\centering
			\includegraphics<1->[width=9cm]{images/stepRespLagCompExampleComparison.eps}
		\end{figure}
	\end{column}
\end{columns}
\end{frame}

\begin{frame}[c]\frametitle{Compensador en Atraso: Ejemplo - Respuesta Rampa Sistema Compensado}
% \vspace*{-2mm}
\begin{columns}
	\begin{column}{0.3\textwidth}
		\onslide<2->{El sistema compensado en lazo cerrado presenta un mejor seguimiento de la referencia rampa.}
	\end{column}
	\begin{column}{0.7\textwidth}
		\begin{figure}
			\centering
			\includegraphics<1->[width=9cm]{images/rampRespLagCompExampleComparison.eps}
		\end{figure}
	\end{column}
\end{columns}
\end{frame}

\section{Compensación en Adelanto-Atraso Usando la Respuesta en Frecuencia}
\begin{frame}[<+->]\frametitle{Compensador en Adelanto - Atraso}
	\begin{equation*}
		G_c(s) = K_c  \underbrace{\left(\frac{s+\frac{1}{T_1}}{s+\frac{\gamma}{T_1}}\right)}_{\text{Red de Adelanto}}  \underbrace{\left(\frac{s+\frac{1}{T_2}}{s+\frac{1}{\beta T_2}}\right)}_{\text{Red de Atraso}}, \hspace*{3mm} (\gamma > 1, \beta > 1)
	\end{equation*}
	\onslide<2->{El compensador se forma por una red de adelanto en cascada con una red de atraso.\\}
	\vspace{2mm}
	\onslide<3->{Ceros del compensador: $s = -1/T_1$, $s = -1/T_2$.\\}
	\vspace{2mm}
	\onslide<4->{Polos del compensador: $s = -\gamma/T_1$, $s = -1/\beta T_2$.\\}
	\vspace{2mm}
	\onslide<5->{Con frecuencia se asume $\gamma = \beta$.}
\end{frame}

\begin{frame}[c]\frametitle{Compensador en Adelanto - Atraso}
	\begin{columns}
		\begin{column}{0.4\textwidth}
			\onslide<1->{Diagrama de Bode del compensador para $K_c = 1$, $\gamma = \beta = 10$, $T_2 = 10 T_1$.\\}
			\vspace{2mm}
			\onslide<2->{Frecuencia central (Fase igual a cero): $\omega_1 = \frac{1}{\sqrt{T_1 T_2}}$.\\}
			\vspace{2mm}
			\onslide<3->{$0 < \omega < \omega_1$: Compensador de atraso.\\}
			\vspace{2mm}
			\onslide<4->{$\omega_1 < \omega < \infty$: Compensador de adelanto.}
		\end{column}	
		\begin{column}{0.6\textwidth}
		\begin{figure}
			\centering
			\includegraphics<2->[width=8.5cm]{images/bodeLagLeadComp.eps}
		\end{figure}
		\end{column}	
	\end{columns}
\end{frame}

\begin{frame}[c]\frametitle{Compensador en Adelanto - Atraso: Procedimiento de Diseño}
	\begin{enumerate}
		\item<2-> Asumir la forma del compensador en adelanto - atraso como:
		\begin{equation*}
			G_c(s) = K_c \frac{\left(s+\frac{1}{T_1}\right)\left(s+\frac{1}{T_2}\right)}{\left(s+\frac{\beta}{T_1}\right)\left(s+\frac{1}{\beta T_2}\right)}, \hspace*{3mm} \beta > 1
		\end{equation*}
		\item<3-> Calcular el valor de $K_c$ necesario para satisfacer el requerimiento de la constante de error estático de velocidad.
		\seti
	\end{enumerate}		
\end{frame}

\begin{frame}[<+->]\frametitle{Compensador en Adelanto - Atraso: Procedimiento de Diseño}
	\begin{enumerate}
		\conti
		\item Obtener el diagrama de bode del sistema no compensado con ganancia ajustada y determinar el márgen de fase. Si es insuficiente continuar con el procedimiento.
		\item Seleccionar una frecuencia de cruce de ganancia deseada $\omega_c'$.
		\item Ubicar la frecuencia de quiebre del cero de la red de atraso $\omega_{z2} = 1/T_2$ una década por debajo de la nueva frecuencia de cruce de ganancia $\omega_c'$.
		\seti
	\end{enumerate}		
\end{frame}

\begin{frame}[c]\frametitle{Compensador en Adelanto - Atraso: Procedimiento de Diseño}
	\begin{enumerate}
		\conti
		\item \onslide<1->{Recordando la relación entre el ángulo de adelanto de fase $\phi_m$ y el parámetro $\alpha$:
		\begin{equation*}
			\sin \phi_m = \frac{1-\alpha}{1+\alpha}
		\end{equation*}}
		\onslide<2->{se sustituye $\alpha = 1/\beta$ obteniendo:
		\begin{equation*}
			\sin \phi_m = \frac{\beta - 1}{\beta + 1}
		\end{equation*}}
		\onslide<3->{A partir del requerimiento de márgen de fase, calcular el valor correspondiente de $\beta$.}
		\item<4-> Calcular la frecuencia de quiebre del polo de la red de atraso como $\omega_{p2} = 1/(\beta T_2)$.
		\seti
	\end{enumerate}		
\end{frame}

\begin{frame}[<+->]\frametitle{Compensador en Adelanto - Atraso: Procedimiento de Diseño}
	\begin{enumerate}
		\conti
		\item Calcular la magnitud del sistema no compensado con ganancia ajustada en la frecuencia de cruce de ganancia deseada.
		\item Asumiendo que el compensador introduce una atenuación del mismo valor, identificar la recta con pendiente 20 db/década que pasa por éste punto. Calcular las frecuencias de cruce con los niveles 0 db y -20 dB. Dichas frecuencias ($\omega_{z1}, \omega_{p1}$) corresponden a los puntos de quiebre (cero y polo) del compensador en adelanto.
		\item Verificar si el sistema de control diseñado satisface los requerimientos. En caso contrario realizar nuevamente la ubicación de los polos y ceros hasta lograr un resultado satisfactorio.
	\end{enumerate}		
\end{frame}

\begin{frame}[c]\frametitle{Compensador en Adelanto - Atraso: Ejemplo}
	Considere el sistema cuya función de transferencia de lazo abierto es:
	\begin{equation*}
		G(s) = \frac{1}{s(s+1)(s+2)}
	\end{equation*}
	Se desea que el error estático de velocidad sea $10\ \text{seg}^{-1}$, el márgen de fase sea \ang{50}, y que el márgen de ganancia sea mayor o igual a 10 dB.
\end{frame}

\begin{frame}[c]\frametitle{Compensador en Adelanto - Atraso: Ejemplo}
	A partir del requerimiento dado por la constante de error estático de velocidad se tiene:
	\begin{align*}
		\onslide<1->{K_v = &\lim_{s \rightarrow 0} s G_c(s)G(s)}
		\onslide<2->{= \lim_{s \rightarrow 0} s K_c \frac{(T_1 s + 1)(T_2 s + 1)}{(\frac{T_1}{\beta}s + 1)(\beta T_2 s + 1)} \frac{1}{s(s+1)(s+2)}}
		\onslide<3->{= \frac{K_c}{2}}
		\onslide<4->{= 10\\}
		\onslide<5->{& \Rightarrow K_c = 20}
	\end{align*}
	\onslide<6->{Se obtiene el diagrama de Bode para el sistema no compensado con ganancia ajustada:
			\begin{equation*}
				G_1(j\omega) = \frac{20}{j\omega(1+j\omega)(2+j\omega)}
			\end{equation*}}
\end{frame}

\begin{frame}[c]\frametitle{Compensador en Adelanto - Atraso: Ejemplo}
	\vspace{-3mm}
	\begin{columns}
		\begin{column}{0.4\textwidth}
			Se obtienen los siguientes márgenes:\\
			\vspace*{2mm}
			\onslide<2->{\textcolor{red}{PM = -\ang{28.1} en 2.43 rad/s}\\}
			\vspace*{2mm}
			\onslide<3->{\textcolor{green}{GM = -10.5 dB en 1.41 rad/s}\\}
			\vspace*{2mm}
			\onslide<4->{El sistema $G_1(s)$ es inestable!}
		\end{column}
		\begin{column}{0.6\textwidth}
			\begin{figure}
				\centering
				\includegraphics<1->[width=8.5cm]{images/marginsExample2.eps}
			\end{figure}
		\end{column}
	\end{columns}
\end{frame}

\begin{frame}[c]\frametitle{Compensador en Adelanto - Atraso: Ejemplo}
	\vspace{3mm}
	\begin{columns}
		\begin{column}{0.5\textwidth}
			\onslide<1->{Para seleccionar $\omega_c'$ se utiliza la frecuencia donde la fase es igual a -\ang{180}:
			\begin{equation*}
				\omega_c' = 1.41 \text{ rad/s}
			\end{equation*}}
			\onslide<2->{La frecuencia de quiebre correspondiente al cero de la red de atraso correponde a:
			\begin{equation*}
				\omega_{z2} = \omega_c'/10 = 0.141 \text{rad/s}
			\end{equation*}}
			\onslide<3->{Por lo tanto $T_2 = 1/\omega_{z2} = 7.0922$.}
		\end{column}
		\begin{column}{0.5\textwidth}
			\onslide<4->{Dado que el margen de fase deseado es \ang{50}, se calcula el valor de $\beta$ como:
			\begin{equation*}
				\sin \phi_m = \frac{\beta-1}{\beta+1} \hspace*{2mm} \Rightarrow \hspace*{2mm} \beta = 7.5486
			\end{equation*}}
			\onslide<5->{Entonces, la frecuencia de quiebre correspondiente al polo de la red de atraso es:
			\begin{equation*}
				\omega_{p2} = \frac{1}{\beta T_2} = 0.0187 \text{ rad/s}
			\end{equation*}}
		\end{column}
	\end{columns}
\end{frame}

\begin{frame}[c]\frametitle{Compensador en Adelanto - Atraso: Ejemplo}
	\onslide<1->{Para obtener los parámetros de la red de adelanto, se observa en el diagrama de Bode que la magnitud del sistema $G_1(s)$ en la frecuencia $\omega_c'$ es igual a 10.5 dB.\\}
	\vspace*{2mm}
	\onslide<2->{Por lo tanto el compensador debe generar una atenuación igual a -10.5 dB en esa misma frecuencia para garantizar el cruce por cero.\\}
	\vspace*{2mm}
	\onslide<3->{Entonces, la curva de magnitud puede aproximarse a una recta con pendiente igual a 20 dB/década que pasa por el punto ($\omega_c',|G_1(j\omega_c')|$) = (1.41,-10.5).\\}
	\vspace*{2mm}
	\onslide<4->{Para calcular las frecuencias de quiebre de la red de adelanto, se calculan los puntos de corte de la recta mencionada con los niveles 0 dB y -20 dB.}
\end{frame}

\begin{frame}[c]\frametitle{Compensador en Adelanto - Atraso: Ejemplo}
	\onslide<1->{Para calcular el punto de corte de la recta con el nivel 0 dB, se escribe la pendiente como:}
	\begin{align*}
		\onslide<2->{20 &= \frac{0 \text{ dB} - (-10.5 \text{dB})}{\log_{10}\omega_{p1} - \log_{10}1.41 \text{ rad/s}}\\}
		\onslide<3->{& \Rightarrow \omega_{p1} = 4.7 \text{ rad/s}}
	\end{align*}
	\onslide<4->{Para calcular el punto de corte de la recta con el nivel -20 dB, se escribe la pendiente como:}
	\begin{align*}
		\onslide<5->{20 &= \frac{-10.5 \text{dB} - (-20 \text{ dB})}{\log_{10}1.41 \text{ rad/s} - \log_{10}\omega_{z1} }\\}
		\onslide<6->{& \Rightarrow \omega_{z1} = 0.47 \text{ rad/s}}
	\end{align*}
\end{frame}

\begin{frame}[c]\frametitle{Compensador en Adelanto - Atraso: Ejemplo}
	Por lo tanto, el compensador obtenido es:
	\begin{align*}
		G_c(s) &= K_c \frac{\left(s+\frac{1}{T_1}\right)\left(s+\frac{1}{T_2}\right)}{\left(s+\frac{\beta}{T_1}\right)\left(s+\frac{1}{\beta T_2}\right)}\\
		 &= 20 \frac{(s+0.47)(s+0.141)}{(s+4.7)(s+0.0187)}
	\end{align*}
\end{frame}

\begin{frame}[c]\frametitle{Compensador en Atraso: Ejemplo - Diagrama de Bode}
\begin{columns}
	\begin{column}{0.35\textwidth}
		\small
		\onslide<1->{Los márgenes obtenidos para el sistema compensado son:\\}
		\vspace*{2mm}
		\onslide<2->{\textcolor{red}{PM = \ang{49.3} en 1.43 rad/s.}\\}
		\vspace*{2mm}
		\onslide<3->{\textcolor{green}{GM = 12.4 dB en 3.46 rad/s.}\\}
		\vspace*{2mm}
		\onslide<4->{Se satisfacen los márgenes especificados en los requerimientos!}
	\end{column}
	\begin{column}{0.65\textwidth}
		\begin{figure}
			\centering
			\includegraphics<1->[width=9cm]{images/lagLeadExample2Comparison.eps}
		\end{figure}
	\end{column}
\end{columns}
\end{frame}

\begin{frame}[c]\frametitle{Compensador en Atraso: Ejemplo - Respuesta Paso Sistema Compensado}
\begin{columns}
	\begin{column}{0.3\textwidth}
		\small
		\onslide<2->{El sistema compensado en lazo cerrado presenta un tiempo de respuesta más rápido, con un sobrepico ligeramente mayor.\\}
		\vspace*{2mm}
		\onslide<3->{Polos de lazo cerrado: \scriptsize \{$-5.7221$, $-0.3386$, $-0.1708$, $-0.7436 \pm 1.8576j$\}.\\}
		\vspace*{2mm}
		\small
		\onslide<4->{Polo dominante muy cercano al eje $j\omega$ $\Rightarrow$ cola de baja amplitud y lenta caída.}
	\end{column}
	\begin{column}{0.7\textwidth}
		\begin{figure}
			\centering
			\includegraphics<1->[width=9cm]{images/stepRespLagLeadCompExample2.eps}
		\end{figure}
	\end{column}
\end{columns}
\end{frame}

\begin{frame}[c]\frametitle{Compensador en Atraso: Ejemplo - Respuesta Rampa Sistema Compensado}
\begin{columns}
	\begin{column}{0.3\textwidth}
		\onslide<2->{El sistema compensado en lazo cerrado presenta un mejor seguimiento de la referencia rampa.}
	\end{column}
	\begin{column}{0.7\textwidth}
		\begin{figure}
			\centering
			\includegraphics<1->[width=9cm]{images/rampRespLagLeadCompExample2.eps}
		\end{figure}
	\end{column}
\end{columns}
\end{frame}

\begin{frame}[<+->]\frametitle{Conclusiones}
\begin{itemize}
	\item Compensación en adelanto: usada para mejorar los márgenes de estabilidad. Se logra gracias a la contribución de ángulo de fase en adelanto.
	\item Compensación en atraso: usada para mejorar el desempeño de estado estacionario. Se logra debido a su atenuación en altas frecuencias.
	\item Es posible que en algunos casos tanto el compensador en adelanto como el de atraso satisfagan los requerimientos.
	\item Con la compensación en adelanto se puede lograr una frecuencia de cruce de ganancia mayor que la que es posible con la compensación en atraso.
	\item El ancho de banda de un sistema con compensación en adelanto siempre es mayor que el de un sistema con compensación en atraso.
\end{itemize}
\end{frame}

\begin{frame}[<+->]\frametitle{Conclusiones}
\begin{itemize}
	\item Si se requiere gran ancho de banda o respuesta rápida $\Rightarrow$ utilizar un compensador en adelanto.
	\item Si hay presencia de ruido un gran ancho de banda no es deseable $\Rightarrow$ utilizar un compensador en atraso.
	\item La compensación en adelanto puede generar señales grandes $\rightarrow$ no es deseable, puede haber saturación.
	\item Compensación en atraso $\rightarrow$ introduce un polo-cero cerca al origen lo cual genera una cola lenta de baja amplitud en la respuesta de tiempo.
	\item Compensador adelanto-atraso $\rightarrow$ permite incrementar la ganancia en baja frecuencia (mejora el desempeño de estado estacionario) a la vez que se aumenta el ancho de banda o los márgenes de estabilidad. 
\end{itemize}
\end{frame}

\end{document}