
\documentclass[11pt, spanish]{article}
\usepackage[T1]{fontenc}
\usepackage[utf8]{inputenc}
\usepackage{amsmath}
\usepackage{amssymb}
% \usepackage{hyperref}
\usepackage{graphicx}
\usepackage{tikz}
\usetikzlibrary{shapes,arrows,positioning,calc}
\usepackage{geometry}
\geometry{
	left=15mm,
	right=15mm,
	top=20mm,
	bottom=20mm,
}

\title{Controles - Taller Clase 6}
\author{Pontificia Universidad Javeriana, Bogotá\\Profesor: Ing. Gerardo Becerra, Ph.D.}
\date{Marzo 18 de 2020}

\begin{document}
	\maketitle

\begin{enumerate}

	\item Considere la función de transferencia de lazo abierto
	\begin{equation*}
		KG(s) = \frac{K}{s(s+2)(s^2+4s+5)}
	\end{equation*}
	\begin{itemize}
		\item Realice un bosquejo del lugar de las raíces. Verifique el resultado usando \texttt{rlocus}.
		\item Calcule la ubicación de los polos dominantes cuando $K = 6.5$.
		\item Para los polos dominantes encontrados, calcule el tiempo de establecimiento y el sobrepico para una entrada paso. Verifique los resultados con una simulación.
	\end{itemize}

	\item Un sistema de control tiene la siguiente función de transferencia de lazo abierto:
		\begin{equation*}
			KG(s) = \frac{K(s+2.5)}{(s^2+2s+2)(s^2+4s+5)}
		\end{equation*}
		\begin{itemize}
			\item Realice un bosquejo del lugar de las raíces. Verifique el resultado usando \texttt{rlocus}.
			\item Encuentre la ganancia $K$ que resulta en polos dominantes con un factor de amortiguamiento de 0.707.
			\item Encuentre el porcentaje de sobrepico y tiempo de pico para la ganancia $K$ calculada. Verifique el resultado usando una simulación.
		\end{itemize}

	\item Un sistema de control tiene la siguiente función de transferencia de lazo abierto:
		\begin{equation*}
			KG(s) = \frac{K(s+1)}{s(s-1)(s+4)}
		\end{equation*}
		\begin{itemize}
			\item Determine el rango de estabilidad de $K$.
			\item Realice un bosquejo del lugar de las raíces. Verifique el resultado usando \texttt{rlocus}.
			\item Determine el máximo valor de $\zeta$ de las raíces complejas estables.
		\end{itemize}

	\item Considere la planta caracterizada por la función de transferencia
		\begin{equation*}
			G(s) = \frac{(s+3)}{(s-1)(s+2)(s+5)}
		\end{equation*}
		Se desea diseñar un sistema de control que cumpla con las siguientes especificaciones ante una entrada paso unitaria:
		\begin{itemize}
			\item Porcentaje de sobrebico: 10\%.
			\item Tiempo de establecimiento: 10 s.
			\item Error de estado estacionario: 0.
		\end{itemize}
		Diseñe un controlador PID usando el lugar de las raíces por el método de aproximación de polos dominantes. Verifique el resultado usando simulaciones.
 	
\end{enumerate}

\end{document}